\documentclass[]{article}
\usepackage{parskip} %Removes paragraph line indents
\usepackage{tabu}
\usepackage{booktabs,array}


\makeatletter
\newcolumntype{V}[1]{>{\topsep=0pt\@minipagetrue}p{#1}<{\vspace{-\baselineskip}}}
\makeatother
\newcommand{\command}[1]{\texttt{\string#1}}


\setlength{\parindent}{0pt} % just for the example

%opening
\title{Group Assignment\\ \ \\Compilers\\ \ \\}
\author{Oliver Mikkel Winther Cristensen (klw741) \\ \ \\Mathias Brøndum Leroul (swn848)\\ \ \\Sebastian O. Jensen (gjx653)\\ \ \\}
\date{Turn-in date: 18-12-2015}

\begin{document}

\maketitle
\newpage

\textbf{Contents of the Report}\\
It is largely up to you to decide what you think is important to include in the report, as long as
the following requirements are met:
Your report should justify all your changes to the compiler modules, in particular, the lexer,
parser, interpreter, type checker, machine-code generator, and the optimization modules. All
major design decisions should be presented and justified.
When evaluating your work, the main focus will be on verifying that your implementation
of the language is correct. While we do not put particular emphasis on compiler optimizations
in this course, we will also evaluate the quality of the generated code: if there are obvious
inefficiencies that could have been easily solved you will be penalized, as they testify either
wrong priorities or lack of understanding.
You should not include the whole compiler in your report, but you must include the parts
that were either added, i.e. new code, or substantially modified. Add them as code listings, and
use the appendix if they get too big. Ideally, we should not need to read your source code.
Your report should describe whether the compilation and execution of your input/test (FASTO)
programs results in the correct/expected behavior. If it does not, try to explain why this is. In
addition, (i) it must be assessed to what extent the delivered test programs cover the language
features, and (ii) if the implementation deviates from the correct/expected behavior than the test
program(s) should illustrate the implementation shortcomings to your best extent.
Known shortcomings in type checking and machine-code generation must be described, and,
whenever possible, you need to make suggestions on how these might be corrected.
The report should not exceed this document in size, and should have an appropriate level of
detail. You might be penalized if your report includes too many irrelevant details.

\section{Introduction}
	We're given the grand task of 
\section{Implementation}
	Each implementation step is presented in the order, which the the code is passed through: Lexer -> Parser -> Interpreter -> Type Checker -> MIPS Code Generator
	
	\subsection*{Notes}
		\begin{itemize}
			\item Implemented Boolean Litterals: true and false.
			\subitem Added test files bool.fo, bool.in, bool.out
		\end{itemize}
	
\subsection{True/False}

	In the first task, we're given the sub-task to implement the boolean operators True and False. We start out by creating the following test files:\\
	
\begin{center}	
	\begin{tabular}{lV{280pt}}
		\toprule
		File name & Code\\
		\midrule
		\command{bool.fo} &
		\begin{verbatim}
			fun int boolTest(int a) = if a == 1
			then 1 else 0
			
			fun int main() =
			    let a = read(int) in
			    write(boolTest(a))
			  
		\end{verbatim}
		\\
		\command{bool.in} &
		\begin{verbatim}
			1
			
		\end{verbatim}
		\\
		\command{bool.out} &
		\begin{verbatim}
			1
		\end{verbatim}
		\\
		\bottomrule \\
	\end{tabular}
\end{center}


Then we implement the True/False token in our parser as well as add True and False to the expression list\\

\begin{center}	
	\begin{tabular}{lV{280pt}}
		\toprule
		File name & Code\\
		\midrule
		\command{Parser.grm} &
		\begin{verbatim}
			%token <(int*int)> TRUE FALSE
			
		\end{verbatim}
		\\
		\command{Parser.grm} &
		\begin{verbatim}
			| TRUE           { Constant (BoolVal true, $1)}
			| FALSE          { Constant (BoolVal false, $1)}
		\end{verbatim}
		\\
		\bottomrule \\
	\end{tabular}
\end{center}

And finally we add the boolean constants to our MIPS code generator\\

\begin{center}	
	\begin{tabular}{lV{280pt}}
		\toprule
		File name & Code\\
		\midrule
		\command{CodeGen.sml} &
		\begin{verbatim}
			| Constant (BoolVal b, pos) => 
			    if b then
			        [ Mips.LI (place, makeConst 1) ] 
			    else
			        [ Mips.LI (place, makeConst 0) ]
		\end{verbatim}
		\\
		\bottomrule \\
	\end{tabular}
\end{center}

\subsection{Multiplication and Division}
Moving on to the implementation of multiplication and division, we again start out with creating test files.\\

\begin{center}	
	\begin{tabular}{lV{280pt}}
		\toprule
		File name & Code\\
		\midrule
		\command{Mult.fo} &
		\begin{verbatim}
			fun int Mult(int a) = a * 2
			
			fun int main() =
			    let a = read(int) in
			    write(Mult(a))
		
		\end{verbatim}
		\\
		\command{Mult.in} &
		\begin{verbatim}
			4
		
		\end{verbatim}
		\\
		\command{Mult.out} &
		\begin{verbatim}
			8
			
		\end{verbatim}
		\\
		\command{Div.fo} &
		\begin{verbatim}
			fun int Div(int a) = a / 2
			
			fun int main() =
			    let a = read(int) in
			    write(Div(a))
			
		\end{verbatim}
		\\
		\command{Div.in} &
		\begin{verbatim}
			128
		
		\end{verbatim}
		\\
		\command{Div.out} &
		\begin{verbatim}
			64
		\end{verbatim}
		\\
		\bottomrule \\
	\end{tabular}
\end{center}

After creating the test files, we swiftly move on to the lexer

\begin{center}	
	\begin{tabular}{lV{280pt}}
		\toprule
		File name & Code\\
		\midrule
		\command{Lexer.lex} &
		\begin{verbatim}
			| `*`  { Parser.TIMES  (getPos lexbuf) }
			| `/`  { Parser.DIVIDE (getPos lexbuf) }
		\end{verbatim}
		\\
		\bottomrule \\
	\end{tabular}
\end{center}

In the parser, we add TIMES and DIVIDE to the already created "PLUS MINUS DEQ EQ LTH" token. They're given a proper precedence order as well as added to the expression list.

\begin{center}	
	\begin{tabular}{lV{280pt}}
		\toprule
		File name & Code\\
		\midrule
		\command{Parser.grm} &
		\begin{verbatim}
			%token <(int*int)> PLUS MINUS DEQ EQ LTH DIVIDE TIMES
		
		\end{verbatim}
		\\
		\command{Parser.grm} &
		\begin{verbatim}
			%left DIVIDE
			%left TIMES
		
		\end{verbatim}
		\\
		\command{Parser.grm} &
		\begin{verbatim}
			Exp :
			        | Exp DIVIDE Exp { Divide ($1, $3, $2) }
			        | Exp TIMES Exp  { Times ($1, $3, $2) }
		\end{verbatim}
		\\
		\bottomrule \\
	\end{tabular}
\end{center}

Moving on to the type checker, we make sure that we can properly check the used types, when doing multiplication and division

\begin{center}	
	\begin{tabular}{lV{280pt}}
		\toprule
		File name & Code\\
		\midrule
		\command{TypeChecker.sml} &
		\begin{verbatim}
		and checkExp ftab vtab (exp : In.Exp)
		  = case exp of
		    .
		    .
		    .
		    | In.Times (e1, e2, pos)
		      => let val (_, e1_dec, e2_dec) = checkBinOp ftab vtab (pos, Int, e1, e2)
		         in (Int,
		             Out.Times (e1_dec, e2_dec, pos))
		         end
		 
		    | In.Divide (e1, e2, pos)
		      => let val (_, e1_dec, e2_dec) = checkBinOp ftab vtab (pos, Int, e1, e2)
		         in (Int,
		             Out.Divide (e1_dec, e2_dec, pos))
		         end		
		\end{verbatim}
		\\
		\bottomrule \\
	\end{tabular}
\end{center}

We go ahead and implement them in our interpreter

\begin{center}	
	\begin{tabular}{lV{280pt}}
		\toprule
		File name & Code\\
		\midrule
		\command{Interpreter.sml} &
		\begin{verbatim}
			| evalExp ( Times(e1, e2, pos), vtab, ftab ) =
			      let val res1   = evalExp(e1, vtab, ftab)
			          val res2   = evalExp(e2, vtab, ftab)
			      in  case (res1, res2) of
			              (IntVal n1, IntVal n2) => IntVal (n1*n2)
			            | _ => invalidOperands  "Multiplication on non-integral args: " [(Int, Int)] res1 res2 pos
			      end
			  
			| evalExp ( Divide(e1, e2, pos), vtab, ftab ) =
			      let val res1   = evalExp(e1, vtab, ftab)
			          val res2   = evalExp(e2, vtab, ftab)
			      in  case (res1, res2) of
			              (IntVal n1, IntVal n2) => IntVal (n1 div n2)
			            | _ => invalidOperands "Division on non-integral args: " [(Int, Int)] res1 res2 pos
			      end
		\end{verbatim}
		\\
		\bottomrule \\
	\end{tabular}
\end{center}

Finally they're added to the MIPS code generator

\begin{center}	
	\begin{tabular}{lV{280pt}}
		\toprule
		File name & Code\\
		\midrule
		\command{CodeGen.sml} &
		\begin{verbatim}
			fun compileExp e vtable place =
			    case e of
			    .
			    .
			    .
			    | Divide (e1, e2, pos) =>
			        let val t1 = newName "divide_L"
			            val t2 = newName "divide_R"
			            val code1 = compileExp e1 vtable t1
			            val code2 = compileExp e2 vtable t2
			        in code1 @ code2 @ [Mips.DIV (place,t1,t2)]
			        end
			    | Times (e1, e2, pos) =>
			        let val t1 = newName "times_L"
			            val t2 = newName "times_R"
			            val code1 = compileExp e1 vtable t1
			            val code2 = compileExp e2 vtable t2
			        in code1 @ code2 @ [Mips.MUL (place,t1,t2)]
			        end
		\end{verbatim}
		\\
		\bottomrule \\
	\end{tabular}
\end{center}

\subsection{AND and OR}
Moving on to the implementation of the boolean literals, AND and OR, we again start out with creating test files.\\

\begin{center}	
	\begin{tabular}{lV{280pt}}
		\toprule
		File name & Code\\
		\midrule
		\command{and.fo} &
		\begin{verbatim}
			fun bool FAnd(bool a, bool b) = a && b
			
			fun bool main() =
			    let a = read(bool) in
			    let b = read(bool) in
			    let c = write(FAnd(a,a)) in
			    let d = write(FAnd(b,b)) in
			    let e = write(FAnd(a,b)) in
			    write(FAnd(b,a))
		
		\end{verbatim}
		\\
		\command{and.in} &
		\begin{verbatim}
		1
		0
		
		\end{verbatim}
		\\
		\command{and.out} &
		\begin{verbatim}
		truefalsefalsefalse
		
		\end{verbatim}
		\\
		\command{or.fo} &
		\begin{verbatim}
			fun bool orTest(bool a, bool b) = a || b
			
			fun bool main() =
			    let a = read(bool) in
			    let b = read(bool) in
			    let c = write(orTest(b,b)) in
			    let d = write(orTest(a,a)) in
			    let e = write(orTest(a,b)) in
			    write(orTest(b,a))
		
		\end{verbatim}
		\\
		\command{or.in} &
		\begin{verbatim}
		1
		0
		
		\end{verbatim}
		\\
		\command{or.out} &
		\begin{verbatim}
		falsetruetruetrue
		\end{verbatim}
		\\
		\bottomrule \\
	\end{tabular}
\end{center}

After creating the test files, we swiftly move on to the lexer

\begin{center}	
	\begin{tabular}{lV{280pt}}
		\toprule
		File name & Code\\
		\midrule
		\command{Lexer.lex} &
		\begin{verbatim}
			| "&&"  { Parser.AND  (getPos lexbuf) }
			| "||"  { Parser.OR   (getPos lexbuf) }
		\end{verbatim}
		\\
		\bottomrule \\
	\end{tabular}
\end{center}

In the parser, we add AND and OR to the "TRUE FALSE" token as well as added to the expression list.

\begin{center}	
	\begin{tabular}{lV{280pt}}
		\toprule
		File name & Code\\
		\midrule
		\command{Parser.grm} &
		\begin{verbatim}
		%token <(int*int)> TRUE FALSE AND OR
				
		\end{verbatim}
		\\
		\command{Parser.grm} &
		\begin{verbatim}
		Exp :
		        | Exp AND Exp { And ($1, $3, $2) }
		        | Exp OR  Exp { Or  ($1, $3, $2) }
		\end{verbatim}
		\\
		\bottomrule \\
	\end{tabular}
\end{center}

Moving on to the type checker, we make sure that we can properly check the used types, when using the boolean literals AND and OR

\begin{center}	
	\begin{tabular}{lV{280pt}}
		\toprule
		File name & Code\\
		\midrule
		\command{TypeChecker.sml} &
		\begin{verbatim}
			| In.And (e1, e2, pos)
			  => let val (t1, e1') = checkExp ftab vtab e1
			         val (t2, e2') = checkExp ftab vtab e2
			     in case (t1 = t2, t1) of
			             (false, _) => raise Error ("And cannot take "^ ppType t1 ^
			                    "and "^ppType t2, pos)
			           | (true, Array _) => raise Error ("And cannot oporate on arrays", pos)
			           | _ => (Bool, Out.And (e1', e2', pos))
			    end
			| In.Or (e1, e2, pos)
			  => let val (t1, e1') = checkExp ftab vtab e1
			         val (t2, e2') = checkExp ftab vtab e2
			     in case (t1 = t2, t1) of
			             (false, _) => raise Error ("Or cannot take "^ ppType t1 ^
			                    "and "^ppType t2, pos)
			           | (true, Array _) => raise Error ("Or cannot operate on arrays", pos)
			           | _ => (Bool, Out.Or (e1', e2', pos))
			     end
		\end{verbatim}
		\\
		\bottomrule \\
	\end{tabular}
\end{center}

We go ahead and implement them in our interpreter

\begin{center}	
	\begin{tabular}{lV{280pt}}
		\toprule
		File name & Code\\
		\midrule
		\command{Interpreter.sml} &
		\begin{verbatim}
			| evalExp (And (e1, e2, pos), vtab, ftab) =
			      let val res1 = evalExp(e1, vtab, ftab)
			          val res2 = evalExp(e2, vtab, ftab)
			      in  case (res1, res2) of
			            (BoolVal n1, BoolVal n2) => BoolVal (n1 andalso n2)
			          | _ => invalidOperands "OR on non-boolean args: " [(Int, Int)] res1 res2 pos
			      end
			  
			| evalExp (Or (e1, e2, pos), vtab, ftab) =
			      let val res1 = evalExp(e1, vtab, ftab)
			          val res2 = evalExp(e2, vtab, ftab)
			      in  case (res1, res2) of
			            (BoolVal n1, BoolVal n2) => BoolVal (n1 orelse n2)
			          | _ => invalidOperands "AND on non-boolean args: " [(Int, Int)] res1 res2 pos
			      end
		\end{verbatim}
		\\
		\bottomrule \\
	\end{tabular}
\end{center}

Finally we add thr short-circuit AND and OR to the MIPS code generator

\begin{center}	
	\begin{tabular}{lV{280pt}}
		\toprule
		File name & Code\\
		\midrule
		\command{CodeGen.sml} &
		\begin{verbatim}
			| And (e1, e2, pos) =>
			  let val t1 = newName "and_L"
			      val t2 = newName "and_R"
			      val code1 = compileExp e1 vtable t1
			      val code2 = compileExp e2 vtable t2
			      val finish = newName "finish"
			  in  code1 @
			      [ Mips.LI (place,"0")
			      , Mips.BEQ (t1, "0", finish) ] @
			      code2 @
			      [ Mips.BEQ (t2, "0", finish)
			      , Mips.LI (place, "1")
			      , Mips.LABEL finish ]
			  end
			
			| Or (e1, e2, pos) =>
			  let val t1 = newName "or_L"
			      val t2 = newName "or_R"
			      val code1 = compileExp e1 vtable t1
			      val code2 = compileExp e2 vtable t2
			      val finish = newName "finish"
			  in  code1 @
			      [ Mips.LI (place,"1")
			      , Mips.BNE (t1, "0", finish) ] @
			      code2 @
			      [ Mips.BNE (t2, "0", finish)
			      , Mips.LI (place, "0")
			      , Mips.LABEL finish ]
			  end
		\end{verbatim}
		\\
		\bottomrule \\
	\end{tabular}
\end{center}

\section{Results}
\section{Conclusion}

\end{document}
